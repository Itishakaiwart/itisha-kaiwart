\documentclass[11pt]{article}




\usepackage{graphicx}
\graphicspath{{image/}}

\title{my}

\begin{document}
\centering
\huge\textbf{TERM PROJECT}\\
on\\
\Huge"3D Bioprinting"
\

\emph{\large \underline{ submitted by:}}\\
\large Itisha Kaiwart\\
\large Roll No. : 21111024\\

\vspace{0.9cm}
\centering\Large\textsc{NATIONAL INSTITUTE OF TECHNOLOGY, RAIPUR}\\

\begin{figure}[h]
\centering
\includegraphics[scale=1.3]{NITRR.jpg}
\end{figure}

\large \underline{ Under the supervision of:}\\
Dr. Saurabh Gupta


\pagebreak

\tableofcontents
\pagebreak

\section{\Large\centering\texttt{ACKNOWLEDGEMENT}}

\large \flushleft I would like to express my special thanks of gratitude to Dr. Saurabh Gupta who gave me the opportunity to do this assignment on"3D Bioprinting".I came to know about so many things I am really thankful to them .\\



\vspace{1cm}
Secondly I would like to thanks my parents and friends who helped me a lot in finalising this assignment within the limited time frame.

\vspace{3cm}

\hfill
\begin{minipage}[t]{9cm}
\begin{flushright}
Itisha Kaiwart \\
21111024\\
1st Semester , Biomedical Engineering\\
National Institute of Technology, Raipur\\
\end{flushright}
\end{minipage}


\pagebreak 

\section{\Large\texttt{ABSTRACT}}
In this term projct report i have focused on stages of bioprinting , types of bioprinting but very few , limitation and application of bioprinting.

\section{\Large\texttt{KEYWORD}}
3D bioprinting , bioinks , pre bioprinting , bioprinting , post bioprinting , inkjet based bioprinting , laser assisted bioprinting , extrusion based bioprinting , applications and limitations.

\section{\Large\texttt{INTRODUCTION}}
3D bioprinting is a mind boggling technology that emerged in the 21st century, the idea of lab grown tissue could mean the end of testing drug on animals and humans, and it could bethe solution for organ shortage and can end the desparate state of organ donation.
clinical trials today are more expensive and lengthy but with bio printing tissue, new products can be assessed and brought to market mor quickly, all without harming test subjects.
\section{\Large\texttt{PRINCIPLE OF 3D BIOPRINTING}}
The principle of 3D printing is based on the precise placement of biological components, biochemical, and living cell in a layer by layer fashion with the spatial control of the placement of functional constitute onto the fabricated 3D structure. the process of 3D bioprinting is based on three distinct approaches; biomimicry or biomimetics, autonomous self assembly, and mini-tissue building blocks. 
\section{\Large\texttt{WHAT IS BIOINKS ?}}
A bioinks is a hydrogel biomaterial that is suitable for bioprinting with mammalian cells and it provide temporary support to the cell while they produce their own extracellular matrix advance in 3 D printing technology as well as development of new bio-inks have made it possible print complex 3D tissue structure.\\
The bioinks used in the bioprinting process should have the following properties;\\
\begin{itemize}
\item 
\end{itemize}
 
\section{\Large\texttt{STAGES OF 3D BIOPRINTING}}
In general 3D bioprinting include three steps;\\
\begin{itemize}
\item Pre-bioprinting: Includes the generation of Computer Aided Design(CAD) of a tissue or organ of interest.A blueprint of tissue and organ can be generated using medical imaging techniques, such as computer tomography(CT) or magnetic resonance imaging(MRI).
 Then blueprint is then converted into a heterogeneous model describingmaterial and cell composition and distribution.The 3D bioprinting are recreated by reducing the particular prototype to a series of 2D layer.
 \item Bioprinting : The printing process is the simultaneous deposition of cell and biomaterials using computer aided precision deposition techique in a layer by layer fashion.
 \item post-bioprinting: The post-processing is the incubation of the printed tissue construct in a biocreator.   


\begin{figure}[h]
\centering
\includegraphics[scale=1.9]{3D.jpg}
\end{figure}

\end{itemize}

\section{\Large\texttt{TYPES OF BIOPRINTER}}
Bioprinting technologies are mainly divided into three categories:
\begin{itemize}


\item Inkjet Based Bioprinting\\
Inkjet-based bioprinting is a non-contact printing technique in which droplets of dilute solutions are dispensed, driven by thermal, piezoelectric, or microvalve processes.\\
\begin{figure}[h]
\centering
\includegraphics[scale=2.8]{inkjet.jpg}
\end{figure}

\item Extrusion Based Bioprinting\\
 Rapid prototyping (RP), also known as solid freeform fabrication, refer to a series of techniques that manufacture objects through sequential delivery of energy and/or material in a layer-by-layer manner per computer aided design (CAD) data.
\begin{figure}[h]
\centering
\includegraphics[scale=0.6]{ex}
\end{figure}
\item Laser Assisted Bioprinting\\
Laser-assisted bioprinting (LAB) uses a laser as the energy source to deposit biomaterials onto a substrate.
\begin{figure}[h]
\centering
\includegraphics[scale=1]{laser.jpg}
\end{figure}



\end{itemize}
\pagebreak

\section{\Large\texttt{APPLICATION OF BIOPRINTING}}
\begin{itemize}
\item DRUG FORMULATIONS\\
Avoid the inherently time-consuming, labour-intensive and dose-inflexible conventional pharmaceutical manufacturing processes and conduct research and development of personalized therapeutic drug formulations by printing tablets or drug delivery devices with alternating sizes and shapes to realize varying release kinetics.\\
\item bioprinting cancer research\\
 These technologies hold great potential for applications in cancer research. Bioprinted cancer models represent a significant improvement over previous 2D models by mimicking 3D complexity and facilitating physiologically relevant cell-cell and cell-matrix interactions.

\end{itemize}

\section{\Large\texttt{limitation of 3D bioprinting}}
\begin{itemize}
\item  One of the most important challenges in 3D bioprinting is to find suitable printing materials with excellent printability, biocompatibility, desired mechanical and degradation properties for tissue constructs.\\
\item Approval from FDA ; specific guidance from FDA does not yet exist for 3D printing in the drug or biologic domains

\end{itemize}


\pagebreak
\section{\Large\texttt{REFERENCE}}
www.news-medical.net\\
www.reseachgate.net\\

 https://microbenotes.com/3d-bioprinting\\
 
https://www.analyticssteps.com/blogs/3d-bioprinting-applications-advantages-and-disadvantages

medical futurist




\end{document}